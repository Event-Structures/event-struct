\subsection{Event Structure of Register Machine}

In this section we present alternative operational semantics
which constructs a prime event structure encoding a 
behavior of the register machine. 

The event structure is constructed incrementally 
in a step-by-step fashion by adding a single event 
on each step. In order to generate a new event 
on each step we require that events behave as \emph{identifiers}.  

\begin{definition}
  We say that set $\eventSet$ together with strict partial order $\identOrd$
  form an identifier set if:
  \begin{itemize}
    \item there exist distinguished initial identifier $e_0 \in \eventSet$;
    \item there exist function $\fresh : \eventSet \fun \eventSet$ which 
      generates a new fresh identifier, \sth
      $$ \forall e \in \eventSet \ldotp e \identOrd \fresh(e) $$
  \end{itemize}
\end{definition}

Following the theory of axiomatic weak memory models~\cite{\todo{}},
we define the causality relation of the register machine's event structure
as a transitive closure of two relations~---~\emph{program order} 
and \emph{reads-from}, denoted as $\lPO$ and $\lRF$ correspondingly. 

$$ \caOrd \ \defeq \ (\lPO \cup \lRF)^* $$

Program order tracks precedence of events within a single thread. 
Reads-from relation captures the flow of values from 
write events to read events, and ensures that values 
do not appear out of thin-air~\cite{\todo{}}.  

In order to construct $\lPO$ and $\lRF$ incrementally
we represent them via their inverse covering functions. 

\begin{definition}[Covering]
  Let $\leqslant$ be a partial order. 
  Then $lessdot$ is covering relation \wrt $\leqslant$ whenever 
  $x \lessdot y$ is true if and only if $x < y$ and 
  there is no $z$ \sth $x < z$ and $z < y$.
  A (non-deterministic) function $f : A \fun \pwset{A}$ is a covering function if
  its corresponding relation, \ie ${\frel{f} \defeq \set{\tup{x, y} ~|~ y \in f(x)}}$, 
  is a covering relation.
\end{definition}
 
We use the inverse covering function because it is 
more convinient in our setting. Indeed, the semanics adds new event
at each step. Then it is convinient to require that in addition 
the small-step relation is provided with the $\lPO$ and $\lRF$
predecessors of a new event.  

\[\def\arraystretch{1.5}
\begin{array}{rcl|rcl}
 
  \lessdot_{\lPO} &\defeq& {\frel{f_{\lPO}}}^{-1}  & \lessdot_{\lRF} &\defeq& {\frel{f_{\lRF}}}^{-1} \\
  \lPO           &\defeq& \lessdot_{\lPO}^+      & \lRF           &\defeq& \lessdot_{\lRF}       \\ 

\end{array}
\] 

The primitive conflict relation is generated by the $f_{\lPO}$ function.
The two events are considered to be in primitive conflict if they are
not equal and have a commond $\lPO$ predecessor. 
For this definition to work properly, we also need 
to assume that in the event structure each thread has 
a special initial event labeled by a distinguished 
\emph{thread start} label $\tslab{}$. 

$$ e_1 \primcfRel e_2 \iff e_1 \neq e_2 \wedge f_{\lPO}(e_1) = f_{\lPO}(e2) $$

Finally, we need a way to reconcile the event structure
with the states of machine's threads. 
To do so we use a function $\thrdmap : \eventSet \fun \ThrdState$
which maps event to a thread state obtained as a result of
execution of event's side-effect.

Let us consider an example. 
\todo{}

\newcommand{\ESAddEventRule}{{(Add Event)}}
\newcommand{\ESIdleRule}{{(Idle)}}
\newcommand{\ESStoreRule}{{(Store)}}
\newcommand{\ESLoadRule}{{(Load)}}
\newcommand{\ESLoadBotRule}{{(Load-Bottom)}}

\begin{figure*}[t]
\begin{center}

  \AXC{$e = \fresh(\first(\eventSeq))$}
  \AXC{$e_{\lPO} \in \eventSeq$}
  \AXC{$e_{\lRF} \in \eventSeq$}
  \RightLabel{\ESAddEventRule}
  \TIC{$\tup{\eventSeq, \lLAB, \lfPO, \lfRF}
        \esstep{\tup{e, \ell, e_{\lPO}, e_{\lRF}}}
        \tup{e :: \eventSeq, \updmap{\lLAB}{e}{\ell}, \updmap{\lfPO}{e}{e_{\lPO}}, \updmap{\lfRF}{e}{e_{\lRF}}}$}
  \DisplayProof

  \rulevspace

  \AXC{$e \in S.\eventSeq$}
  \AXC{$\thrdst = \appmap{\thrdmap}{e}$}
  \AXC{$P \vdash \thrdst \thrdstep{\eps} \thrdst'$}
  \RightLabel{\ESIdleRule}
  \TIC{$P \vdash \tup{S, \thrdmap} \fullstep{\eps} \tup{S, \updmap{\thrdmap}{e}{\thrdst'}}$}
  \DisplayProof

  \rulevspace

  \AXC{$\thrdst = \appmap{\thrdmap}{e_{\lPO}}$}
  \AXC{$P \vdash \thrdst \thrdstep{\ell} \thrdst'$}
  \AXC{$S \esstep{\tup{e, \ell, e_{\lPO}, \bot}} S'$}
  \AXC{$l = \wlab{x}{v}$}
  \RightLabel{\ESStoreRule}
  \QIC{$P \vdash \tup{S, \thrdmap} \fullstep{\tup{e, \ell, e_{\lPO}, \bot}} \tup{S', \updmap{\thrdmap}{e}{\thrdst'}}$}
  \DisplayProof

  \rulevspace

  \AXC{$\thrdst = \appmap{\thrdmap}{e_{\lPO}}$}
  \AXC{$P \vdash \thrdst \thrdstep{\ell} \thrdst'$}
  \AXC{$S \esstep{\tup{e, \ell, e_{\lPO}, e_{\lRF}}} S'$}
  \AXC{$\neg (e_{\lPO} \cfRel e_{\lRF})$}
  \AXC{$l = \rlab{x}{v}$}
  \noLine
  \UIC{$\appmap{\lLAB}{e_{\lRF}} = \wlab{x}{v}$}
  \RightLabel{\ESLoadRule}
  \QuinaryInfC{$P \vdash \tup{S, \thrdmap} \fullstep{\tup{e, \ell, e_{\lPO}, e_{\lRF}}} \tup{S', \updmap{\thrdmap}{e}{\thrdst'}}$}
  \DisplayProof

  \rulevspace

  \AXC{$\thrdst = \appmap{\thrdmap}{e_{\lPO}}$}
  \AXC{$P \vdash \thrdst \thrdstep{\ell} \thrdst'$}
  \AXC{$S \esstep{\tup{e, \ell, e_{\lPO}, \bot}} S'$}
  \AXC{$l = \rlab{x}{\bot}$}
  \RightLabel{\ESLoadBotRule}
  \QIC{$P \vdash \tup{S, \thrdmap} \fullstep{\tup{e, \ell, e_{\lPO}, \bot}} \tup{S', \updmap{\thrdmap}{e}{\thrdst'}}$}
  \DisplayProof

  
  \caption{Semantics of register machine event structure}
  \label{fig:eventstruct-sem}

\end{center}
\end{figure*}


The rules of operational semantics constructing 
the event structure are presented on~\cref{fig:eventstruct-sem}.
The first auxiliary rule \ESAddEventRule adds a new event, sets its 
label, $\lPO$ and $\lRF$ predecessors. 
The \ESIdleRule handles the case when a thread of 
the register machine performs an internal step 
without any side effect. 
It chooses an event $e$ together with 
the thread state $s$ corresponding to it
and performs one step reduction to new state $s'$.
It then updates the mapping of events to thread states.   
The last two rules \ESStoreRule and \ESLoadRule
correspond to store and load perfromed by some thread.  
Similarly to \ESIdleRule, an event $e_{\lPO}$ is selected
and one reduction is performed from the corresponding thread state $s$.
Unlike the \ESIdleRule case, however, a new event $e$ is also generated.
In case of \ESLoadRule additionally an event $e_{\lRF}$ is selected,
such that it has a write label matching the read label of the new event.    

The following theorem asserts that the event structure built this way
indeed satisfies the axioms of the prime confusion free event structure.  

\begin{theorem}
  The tuple $\tup{E, \caOrd, \primcfRel}$, where $\caOrd$ and $\primcfRel$
  are defined as described above, forms prime confusion-free event structure.
\end{theorem}

%% I can't use `proof` here, sorry.  
% \begin{proof}
% \end{proof}

\todo{Sketch the proof if we'll have space and time}

