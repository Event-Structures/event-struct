\subsection{Event Structure of Register Machine}

In this section we present alternative operational semantics
which constructs a prime event structure encoding a 
behavior of the register machine. 

The event structure is constructed incrementally 
in a step-by-step fashion by adding a single event 
on each step. In order to generate a new event 
on each step we require that events behave as \emph{identifiers}.  

\begin{definition}[Identifiers set]
  We say that set $\eventSet$ together with strict partial order $\identOrd$
  form an identifier set if:
  \begin{itemize}
    \item there exist distinguished initial identifier $e_0 \in \eventSet$;
    \item there exist function $\fresh : \eventSet \fun \eventSet$ which 
      generates a new fresh identifier, \sth
      $$ \forall e \in \eventSet \ldotp e \identOrd \fresh(e) $$
  \end{itemize}
\end{definition}

Following the theory of axiomatic weak memory models~\cite{\todo{}},
we define the causality relation of the register machine's event structure
as a transitive closure of two relations~---~\emph{program order} 
and \emph{reads-from}, denoted as $\lPO$ and $\lRF$ correspondingly. 

$$ \caOrd \ \defeq \ (\lPO \cup \lRF)^* $$

Program order tracks precedence of events within a single thread. 
Reads-from relation captures the flow of values from 
write events to read events, and ensures that values 
do not appear out of thin-air~\cite{\todo{}}.  

In order to construct $\lPO$ and $\lRF$ incrementally
we represent them via their inverse covering functions. 

\begin{definition}[Covering]
  Let $\leqslant$ be a partial order. 
  Then $lessdot$ is covering relation \wrt $\leqslant$ whenever 
  $x \lessdot y$ is true if and only if $x < y$ and 
  there is no $z$ \sth $x < z$ and $z < y$.
  A (non-deterministic) function $f : A \fun \pwset{A}$ is a covering function if
  its corresponding relation, \ie ${\frel{f} \defeq \set{\tup{x, y} ~|~ y \in f(x)}}$, 
  is a covering relation.
\end{definition}
 
We use the inverse covering function because it is 
more convinient in our setting. Indeed, the semanics adds new event
at each step. Then it is convinient to require that in addition 
the small-step relation is provided with the $\lPO$ and $\lRF$
predecessors of a new event.  

\[\def\arraystretch{1.5}
\begin{array}{rcl|rcl}
 
  \lessdot_{\lPO} &\defeq& {\frel{f_{\lPO}}}^{-1}  & \lessdot_{\lRF} &\defeq& {\frel{f_{\lRF}}}^{-1} \\
  \lPO           &\defeq& \lessdot_{\lPO}^+      & \lRF           &\defeq& \lessdot_{\lRF}       \\ 

\end{array}
\] 

The primitive conflict relation is generated by the $f_{\lPO}$ function.
The two events are considered to be in primitive conflict if they are
not equal and have a commond $\lPO$ predecessor. 
For this definition to work properly, we also need 
to assume that in the event structure each thread has 
a special initial event labeled by a distinguished 
\emph{thread start} label $\tslab{}$. 

$$ e_1 \primcfRel e_2 \iff e_1 \neq e_2 \wedge f_{\lPO}(e_1) = f_{\lPO}(e2) $$

Finally, we need a way to reconcile the event structure
with the states of machine's threads. 
To do so we use a function $\thrdmap : \eventSet \fun \ThrdState$
which maps event to a thread state obtained as a result of
execution of event's side-effect.

Let us consider an example. 
\todo{}

The rules of operational semantics constructing 
the event structure are presented on~\cref{fig:eventstruct-sem}.

\newcommand{\ESAddEventRule}{{(Add Event)}\xspace}
\newcommand{\ESIdleRule}{{(Idle)}\xspace}
\newcommand{\ESStoreRule}{{(Store)}\xspace}
\newcommand{\ESLoadRule}{{(Load)}\xspace}
\newcommand{\ESLoadBotRule}{{(Load-Bottom)}\xspace}

\begin{figure*}[t]
\begin{center}

  \AXC{$e = \fresh(\first(\eventSeq))$}
  \AXC{$e_{\lPO} \in \eventSeq$}
  \AXC{$e_{\lRF} \in \eventSeq$}
  \RightLabel{\ESAddEventRule}
  \TIC{$\tup{\eventSeq, \lLAB, \lfPO, \lfRF}
        \esstep{\tup{e, \ell, e_{\lPO}, e_{\lRF}}}
        \tup{e :: \eventSeq, \updmap{\lLAB}{e}{\ell}, \updmap{\lfPO}{e}{e_{\lPO}}, \updmap{\lfRF}{e}{e_{\lRF}}}$}
  \DisplayProof

  \rulevspace

  \AXC{$e \in S.\eventSeq$}
  \AXC{$\thrdst = \appmap{\thrdmap}{e}$}
  \AXC{$P \vdash \thrdst \thrdstep{\eps} \thrdst'$}
  \RightLabel{\ESIdleRule}
  \TIC{$P \vdash \tup{S, \thrdmap} \fullstep{\eps} \tup{S, \updmap{\thrdmap}{e}{\thrdst'}}$}
  \DisplayProof

  \rulevspace

  \AXC{$\thrdst = \appmap{\thrdmap}{e_{\lPO}}$}
  \AXC{$P \vdash \thrdst \thrdstep{\ell} \thrdst'$}
  \AXC{$S \esstep{\tup{e, \ell, e_{\lPO}, \bot}} S'$}
  \AXC{$l = \wlab{x}{v}$}
  \RightLabel{\ESStoreRule}
  \QIC{$P \vdash \tup{S, \thrdmap} \fullstep{\tup{e, \ell, e_{\lPO}, \bot}} \tup{S', \updmap{\thrdmap}{e}{\thrdst'}}$}
  \DisplayProof

  \rulevspace

  \AXC{$\thrdst = \appmap{\thrdmap}{e_{\lPO}}$}
  \AXC{$P \vdash \thrdst \thrdstep{\ell} \thrdst'$}
  \AXC{$S \esstep{\tup{e, \ell, e_{\lPO}, e_{\lRF}}} S'$}
  \AXC{$\neg (e \cfRel e_{\lRF})$}
  \AXC{$l = \rlab{x}{v}$}
  \noLine
  \UIC{$\appmap{\lLAB}{e_{\lRF}} = \wlab{x}{v}$}
  \RightLabel{\ESLoadRule}
  \QuinaryInfC{$P \vdash \tup{S, \thrdmap} \fullstep{\tup{e, \ell, e_{\lPO}, e_{\lRF}}} \tup{S', \updmap{\thrdmap}{e}{\thrdst'}}$}
  \DisplayProof

  \rulevspace

  \AXC{$\thrdst = \appmap{\thrdmap}{e_{\lPO}}$}
  \AXC{$P \vdash \thrdst \thrdstep{\ell} \thrdst'$}
  \AXC{$S \esstep{\tup{e, \ell, e_{\lPO}, \bot}} S'$}
  \AXC{$l = \rlab{x}{\bot}$}
  \RightLabel{\ESLoadBotRule}
  \QIC{$P \vdash \tup{S, \thrdmap} \fullstep{\tup{e, \ell, e_{\lPO}, \bot}} \tup{S', \updmap{\thrdmap}{e}{\thrdst'}}$}
  \DisplayProof

  
  \caption{Semantics of register machine event structure}
  \label{fig:eventstruct-sem}

\end{center}
\end{figure*}


\eupp{Continue from here}

As we saw earlier in Step by Step Example, our event structure contains functions such as $\frf$ and $\fpred$. The events themselves are elements of some set $\mathcal{E}$ (set of identifiers), which must satisfy the following properties
We can fix any $\mathcal{E}$ (for example $\mathbb{N}$) and theat programm's events as a finite subset of $\mathcal{E}$

Let $E$ be the set of all program events, then we have
  $$\ffun{lab} : E \to \lb $$
where $\lb$ is the label's set (such that $\wlab{x}{5}, \rlab{y}{6}$ ThreadStart, etc.)

We will say that $e$ is a "read" -event with location x and value i if $\ffun{lab}(e) = \rlab{x}{i}$, similarly to "write" -event

As a result, we formally build a four
\begin{definition}[Execution Event Structure]
  $$exec\_event\_structure \ \triangleq \ \langle E, \ \ffun{lab}, \ \fpred, \ \frf \rangle$$
  where
  \begin{itemize}
    \item $E$ ~--- finite subset of some identifiers set $\mathcal{E}$
    \item $\fpred$ ~--- function that returns a predsessor (w.r.t. the program order) of event
    \item $\frf$ ~--- function that takes "read"-event $e$ and returns a "write"-event from witch $e$ reads
    \item $\ffun{lab} : E \to \lb$ ~--- function that returns event's label
  \end{itemize}
  Also we require
  \begin{itemize}
    \item $\forall e \in E, \ \fpred(e) \leqslant_E e$ and $\frf(e) \leqslant_E e$
    \item $\forall e_1 \ e_2 \in E, \ e_2 = \frf(e_1) \ \Leftrightarrow \ e_1$ is a "read"-event and $e_2$ is a "write"-event with the same location and value
  \end{itemize}
\end{definition}
Further, for simplicity, we will identify identifiers with their labels, and Execution Event Structure with their set of identifiers

Now our goal is to define a prime event structures in these terms. To do this, one need to come up with a causality relation $\leqslant$ and a conflict relation $\#$

Let's start with the causality. In our article, we will assume that the event $e_1$ entails the event $e_2$ if
\begin{itemize}
  \item Programm instruction, that corresponds to event $e_1$ states before the programm instruction, that corresponds to $e_2$
  \item Or $\ e_2$ is the "read"-event and it reads from "write"-event $e_1$
\end{itemize}
Or formally:
\begin{itemize}
  \item $\fpred(e_2) \ =\ e_1$
  \item Or $\ \frf(e_2) \ =\ e_1$
\end{itemize}
Any function $f$ can be treated as a relation
  $$ a \ f \ b \ \triangleq\ f(b) \ = \ a$$
With the notation above, we define

\begin{center}
\begin{definition}[causality relation]
  $$\leqslant \ \triangleq \ (\fpred \cup \frf)^* $$
\end{definition}
\end{center}

Note that the set $E \subset \mathcal{E}$ already contains the total order $ \leqslant_\mathcal{E} $. One can note that the resulting casuality relation is its suborder [TODO:see Well-foundedness and monads]

\begin{center}
\begin{lemma}
  If $\leqslant$ is a casuality order of $E \subset \mathcal{E}$ then $\leqslant \ \subset \ \leqslant_\mathcal{E}$
\end{lemma}
\end{center}

Now let's define a conflict relation. Looking at the conflict relation in \cref{es_def}, we can establish the following fact:
\begin{center}
\begin{lemma}\label{lma_confl}
  If $\ e_1 \leqslant e_2, \ e_3 \leqslant e_4$ and $e_1 \# e_3$ holds then $e_2 \# e_4$
\end{lemma} 
\end{center}
That is, the conflict relation must be "dragged" through the $\leqslant$ relation. Our intuition will be as follows: we, first, define a "minimal" conflict relation, by closing which under the consistency property (\cref{lma_confl}), we finally get a full relation $\#$. 

So where do conflicts come from?

From the step-by-step example, we see that if events have a common predsessor (w.r.t. the $\fpred$ relation), then they occur in two different program executions and, accordingly, are conflicting
\begin{definition}[immediate conflict relation]
  $$ e_1 \, \ffun{icf} \, e_2 \ \triangleq\ \exists e, \, e = \fpred(e_1) = \fpred(e_2)  $$
\end{definition}
Next, one can define the conflict relation as a consistent closure of $\ffun{icf}$
\begin{definition}[conflict relation]
  $$ e_1 \, \# \, e_2 \ \triangleq\ \exists e_3\, e_4, e_3 \, \ffun{icf} \, e_4 \ and \ e_3 \leqslant e_1, \, e_4 \leqslant e_2 $$
\end{definition}
However, there is one problem with such definition. It is clear that the relation defined in this way turns out to be symmetric and consistent. Will it be irreflexive? It turns out that in the general case, no. Consider the following program:
\begin{figure}[h]
  \begin{center}
    \begin{tikzpicture}

      \node (iwy) %[left of = mid, node distance = 1.1cm] 
      at (-1.3, 2.5)
      {$x:=1;$};
    
      \node (irx) [above = -0.5cm of iwy.west, anchor = west] 
      {$r:=x;$};
    
      \node (iwx) %[right of = mid, node distance = 0.8cm] 
       at (0.5, 2.5)
      {$x:=2;$};
    
      \node (iwz) %[right of = mid, node distance = 0.8cm] 
      at (-1.3, 1.55)
     {$y:=r;$};
    
     \node (iww) %[right of = mid, node distance = 0.8cm] 
      at (-1.3, 1.2)
     {$r':=y;$};
    
      \draw ($ (iwy.south east) + (0.1,-1.3) $) -- ($ (iwy.south east) + (0.1,0.6) $);
      \draw ($ (iwx.south west) + (-0.15,-1.3) $) -- ($ (iwx.south west) - (0.15,-0.6) $);
    \end{tikzpicture}
  \end{center}
\end{figure}

Let's build, as in the Step-by-step example, a fragment of $exec\_event\_structure$ corresponding to this program before reading from $y$
\begin{figure}[h]
  \begin{center}
    \begin{tikzpicture}
      \node (ts1) at (-3,1.2) {ThreadStart};
      \node (ts2) at (-1, 1.2) {ThreadStart};
      \node (wxo) at (-3,0.2) {$\wlab{x}{1}$};
      \node (wxt) at (-1,0.2) {$\wlab{x}{2}$};
      \node (rxt) at (-2, -1.2) {$\rlab{x}{2}$};
      \node (rxo) at (-4, -1.2) {$\rlab{x}{1}$};
      \node (wyo) at (-4, -2.2) {$\wlab{y}{1}$};
      \node (wyt) at (-2, -2.2) {$\wlab{y}{2}$};
      \draw[po] (wxo)  --  (ts1);
      \draw[po] (wxt)  --  (ts2);
      \draw[po] (rxt)  --  (wxo);
      \draw[rf] (rxt)  --  (wxt);
      \draw[po] (rxo)  --  (wxo);
      \draw[rf] (rxo)  edge[bend left] (wxo);
      \draw[cf] (rxo)  -- node[above]{$\#$}  (rxt);
      \draw[po] (wyo) -- (rxo);
      \draw[po] (wyt) -- (rxt);
    \end{tikzpicture}
  \end{center}
\end{figure}

One can observe, that our program in the first thread was divided into two independent executions: execution, in which we read from $x$ value 1 (denote by execution (a)) and execution, in which we read from $x$ 2 (execution (b)).

Now we want to add a read-from-$y$ event to our event structure. Let's see what is going to happen if we will add "read"-event, reading from $\wlab{y}{2}$ (see \cref{fig:cf}).
\input{fig/encode_cf.tex}

We have $\rlab{x}{1} \leqslant \rlab{y}{2}$, $ \rlab{x}{2} \leqslant \rlab{y}{2}$ and $\rlab{x}{1} \ \# \ \rlab{x}{2}$. Thus $\rlab{y}{2} \ \# \ \rlab{y}{2}$ holds (see~\cref{lma_confl})

To deal with this program, we need to impose a constraint on the event structure.

We will say that the event structure is consistent if there are no reads from conflicting writes. Formally:
\begin{definition}[consistent Execution Event Structure]
  $$consitent\langle E, \ffun{lab}, \fpred, \frf \rangle \ \triangleq \ \forall e \in E, \ \neg (e \ \# \ \frf e)$$
\end{definition}
It turns out that this is enough for the conflict relation to be irreflexive. %(see \cref{fig:full})
\begin{lemma}
  Let $E$ be a consistent Execution Event Structure (denote as $cexec\_event\_structure$). Then it's conflict relation $\#$ is irreflexive
\end{lemma}
%\input{fig/encode_full.tex} 
So, we see that from the consistent event structure one can get the Prime Event Structure