\subsection{Gentle introduction to Event Structures}

Generally, event structures can model a process of computation. 
There are many modifications of events structures from elementary to general, 
and we are using prime event structures, since it is enough for us to model 
concurrent programs. More precisely, they represent that process by a set of events 
and relations between them: 
\begin{definition}
A prime event structure (PES) is a triple $(E, \leq, \#)$, where
\begin{itemize}
  \item E is a set of events
  \item $\leq$ is a causality relation on E such that 
  \begin{itemize}
    \item $ (E, \leq) $ is a partial order
    \item $\lceil e \rceil := \{ e' : e' < e \}$ is finite for every $e \in E$, 
    i.e. every event is caused by a finite set of events
  \end{itemize}
  \item $\#$ is a conflict relation on $E$ such that
  \begin{itemize}
    \item $\#$ is irreflexive and symmetric
    \item it satisfies a consistency predicate:
    $$\forall e_1, e_2, e_3 \in E \ e_1 \# e_2 \land e_2 \leq e_3 \Rightarrow e_1 \# e_3.$$
  \end{itemize}
\end{itemize}
\end{definition}

Conflict relation is used to model conflicting events like two \texttt{write}'s 
of different values to the same location.
Consistency ensures that if some events are in conflict, then events 
following these events are also in conflict.

\begin{definition}
A configuration of PES $(E, \leq, \#)$ is a subset $X \subseteq E$ such that
\begin{itemize}
  \item it is conflict-free
  \item it is left-closed, i.e. $$\forall e_1, e_2 \in E \ e_2 \in X \land e_1 \leq e_2 \Rightarrow e_1 \in E.$$
\end{itemize}
\end{definition}

Configurations are used to model a history of computation up to a certain 
execution point.

We formally proved the following simple facts about prime event structures:

\begin{lemma}
  A set $\lceil e \rceil$ is conflict-free for any $e \in E$.
\end{lemma}

And a more general one:

\begin{lemma}
  A set $\lceil e \rceil$ is a configuration for any $e \in E$.
\end{lemma}