\subsection{Transitive Closure of Covering Function}

\todo{Sketch the problem of transitive closure termination 
and our solution if we'll have space and time}

% Initially, we used $\mathtt{nat}$ type to model event structures,
% but later we realised that using an arbirary well-founded set as a 
% domain of our model is a good generalisation for our results. 

% However, here arises one problem: we need the operation of 
% reflexive-transitive closure on binary relations over 
% general well-founded type, but $\mathtt{mathcomp}$ doesn't
% have it for them.

% So, we introduce our definition of transitive closure for a 
% decreasing function:

% \begin{definition}
%   Let $f : T \to \mathtt{seq} \ T$ be a decreasing function from 
%   well-founded set $T$ to lists of elements of $T$. Then the 
%   transitive closure of this function is the function 
%   $(\mathtt{suffix} \ f)$ from $T$ to lists of elements 
%   of $T$ such that for any $x : T$ the list $\mathtt{suffix} \ f \ x$
%   contain $f \ x$ and all elements of closures of elements of $f \ x$.
% \end{definition}

% Here, well-foundedness of $T$ guarantees that the process of the 
% calculation of the closure is finite for every element.

% Also, from this definition it is easy to see, that we can easily 
% generate binary relation on $T$, that corresponds to the closure:
% $$x \leq y \Leftrightarrow x \in \mathtt{suffix} \ f \ y.$$

% We also introduce a reflexive-transitive closure definition by 
% addition of element itself to its $\mathtt{suffix}$ list.

% For these definition we proved their correctness, i.e. we proved
% that the relation generated from our definition can be reflected 
% to the traditional Coq's inductive definition of the closure.

% Currently, we are working on the generalisation of the initial 
% function for $\mathtt{seq}$ to an arbitrary function 
% $f : T \to M \ T$, where M is such a monad, that there is a monad 
% morphism from $M$ to the list monad.

% This monad morphism allows us to use $\texttt{in}$ notation for
% monads by passing values to lists.

% For this problem we use $\mathtt{monae}$ Coq library, that 
% contains a hierarchy of monads. Using this library we defined
% monad morphisms for arbitrary monads and its special case for 
% failure monads.
