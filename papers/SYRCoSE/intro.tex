\section{Introduction}

Event structures is a mathematical formalism introduced 
by Winskel~\cite{Winskel:86} as a semantic domain of concurrent programs.
In recent years there is a renewed interest in event structures, 
with the applications of the theory ranging from relaxed memory models%
~\cite{Jeffrey-Riely:LICS16, PichonPharabod-Sewell:POPL16, Chakraborty-Vafeiadis:POPL19}
to model-based mutation testing~\cite{Fellner-al:VMCAI2020}.
% \eupp{perhaps, we could add more citations once the related work is ready.}

The main advantage of event structures
compared to traditional interleaving semantics 
is that they give a more compact and concise 
representation of programs' behaviours.
For example, consider the following code snippet
of a simple parallel program.

\begin{center}
\begin{tikzpicture}
      \node (wxa) %[left of = mid, node distance = 1.1cm] 
        % at (0,1)
        {$x:=1$};

      \node (wxb) [right of = wxa, node distance = 1.3cm] 
        % at (1,1)
        {$x:=2$};

      \node (wxc) [right of = wxb, node distance = 1.3cm] 
        % at (2,1)
        {$x:=3$};

      \node (rx) [below of = wxb, node distance = 0.7cm] 
        % at (1,0)
        {$r:=x$};

      \draw ($(wxa.north east)$) -- ($(wxa.south east)$);
      \draw ($(wxb.north west)$) -- ($(wxb.south west)$);
      \draw ($(wxb.north east)$) -- ($(wxb.south east)$);
      \draw ($(wxc.north west)$) -- ($(wxc.south west)$);
\end{tikzpicture}
\end{center}


Under the interleaving semantics 
it has $3! = 6$ traces with each trace consisting of $4$ events,
as depicted in \cref{fig:intro-traces}.
Events themselves represent atomic side-effects produced by instruction executions.
In our case an event is either a \emph{write} of a value $a$ to a shared variable $x$ denoted as $\wlab{x}{a}$,
or a \emph{read} of a value $a$ from a shared variable $x$ denoted as $\rlab{x}{a}$.  

\begin{figure}[h]
\footnotesize
\begin{center}
\begin{tikzpicture}[xscale=1.5, yscale=0.7]
    \node (wxaa) at (0,3) {$\wlab{x}{1}$};
    \node (wxab) at (0,2) {$\wlab{x}{2}$};
    \node (wxac) at (0,1) {$\wlab{x}{3}$};
    \node (rxa)  at (0,0) {$\rlab{x}{3}$};

    \draw[ca] (wxaa) -- (wxab);
    \draw[ca] (wxab) -- (wxac);
    \draw[ca] (wxac) -- (rxa);

    \draw ($(wxaa.north east) + (0.1,0)$) -- ($(rxa.south east) + (0.1,0)$);


    \node (wxba) at (1,3) {$\wlab{x}{1}$};
    \node (wxbb) at (1,2) {$\wlab{x}{3}$};
    \node (wxbc) at (1,1) {$\wlab{x}{2}$};
    \node (rxb)  at (1,0) {$\rlab{x}{2}$};

    \draw[ca] (wxba) -- (wxbb);
    \draw[ca] (wxbb) -- (wxbc);
    \draw[ca] (wxbc) -- (rxb);

    \draw ($(wxba.north east) + (0.1,0)$) -- ($(rxb.south east) + (0.1,0)$);

    \node (wxca) at (2,3) {$\wlab{x}{2}$};
    \node (wxcb) at (2,2) {$\wlab{x}{1}$};
    \node (wxcc) at (2,1) {$\wlab{x}{3}$};
    \node (rxc)  at (2,0) {$\rlab{x}{3}$};

    \draw[ca] (wxca) -- (wxcb);
    \draw[ca] (wxcb) -- (wxcc);
    \draw[ca] (wxcc) -- (rxc);

    \draw ($(wxca.north east) + (0.1,0)$) -- ($(rxc.south east) + (0.1,0)$);

    \node (wxda) at (3,3) {$\wlab{x}{2}$};
    \node (wxdb) at (3,2) {$\wlab{x}{3}$};
    \node (wxdc) at (3,1) {$\wlab{x}{1}$};
    \node (rxd)  at (3,0) {$\rlab{x}{1}$};

    \draw[ca] (wxda) -- (wxdb);
    \draw[ca] (wxdb) -- (wxdc);
    \draw[ca] (wxdc) -- (rxd);

    \draw ($(wxda.north east) + (0.1,0)$) -- ($(rxd.south east) + (0.1,0)$);

    \node (wxea) at (4,3) {$\wlab{x}{3}$};
    \node (wxeb) at (4,2) {$\wlab{x}{1}$};
    \node (wxec) at (4,1) {$\wlab{x}{2}$};
    \node (rxe)  at (4,0) {$\rlab{x}{2}$};

    \draw[ca] (wxea) -- (wxeb);
    \draw[ca] (wxeb) -- (wxec);
    \draw[ca] (wxec) -- (rxe);

    \draw ($(wxea.north east) + (0.1,0)$) -- ($(rxe.south east) + (0.1,0)$);

    \node (wxfa) at (5,3) {$\wlab{x}{3}$};
    \node (wxfb) at (5,2) {$\wlab{x}{2}$};
    \node (wxfc) at (5,1) {$\wlab{x}{1}$};
    \node (rxf)  at (5,0) {$\rlab{x}{1}$};

    \draw[ca] (wxfa) -- (wxfb);
    \draw[ca] (wxfb) -- (wxfc);
    \draw[ca] (wxfc) -- (rxf);

    % \draw ($(wxfa.north east) + (0.1,0)$) -- ($(rxf.south east) + (0.1,0)$);

\end{tikzpicture}
\caption{}
\label{fig:intro-traces}
\end{center}
\end{figure}

The same information can be encoded in a single 
event structure containing $6$ events in total
(see \cref{fig:intro-es}). 
In the event structure there are two types of edges 
between the events. The grey arrows $e_1 \arrowCA e_2$ 
represent the \emph{causality relation}, a 
partial order reflecting the causal relationship
between the atomic events of computation.
The red edges $e_1 \arrowCF e_2$ represent 
the \emph{conflict relation} which is 
a symmetric and irreflexive relation 
encoding mutually exclusive events.
Each particular trace can be extracted from the event structure
as a linearisation of some \emph{configuration}, 
that is a causally-closed and conflict-free 
subset of events. 

\begin{figure}[h]
    \begin{center}
    \begin{tikzpicture}[xscale=2]
        \node (wxa) at (0,1) {$\wlab{x}{1}$};
        \node (wxb) at (1,1) {$\wlab{x}{2}$};
        \node (wxc) at (2,1) {$\wlab{x}{3}$};
        \node (c)   at (1,0) {$\bullet$};
    
        \node (rxa) at (0,-1) {$\rlab{x}{1}$};
        \node (rxb) at (1, -2) {$\rlab{x}{2}$};
        \node (rxc) at (2,-1) {$\rlab{x}{3}$};
    
        \draw[ca] (wxa) -- (rxc);
        \draw[ca] (wxb) -- (rxb);
        \draw[ca] (wxc) -- (rxa);
    
        \draw[cf] (rxa) -- (rxb) -- (rxc);
        \draw[cf] (rxa) -- (rxc);
    \end{tikzpicture}
    %\caption{Event Structure}
    \label{fig:intro-es}
    \end{center}
    \end{figure}
    

The programming languages theory and formal semantics research communities 
are moving to increase the usage of \emph{proof assistants} 
like Coq, Agda, Isabelle, Arend, and others,
to complement theoretical studies with their mechanisation,
as this process increases the reliability and reproducibility 
of scientific results.
Yet, to the best of our knowledge, there is little work on 
mechanisation of the theory of event structures.
The present report aims to close the gap.

Our goal is to develop a Coq library containing 
a comprehensive set of common definitions, lemmas, 
and tactics that would allow researchers 
to utilise the theory of event structures 
for the needs of their domain.

In this work-in-progress report we sketch 
the common design principles behind our library
and give a concrete example of its usage  
by developing a formal mechanised semantics of simple 
register machine with shared memory.
