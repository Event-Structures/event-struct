\subsection{Register Machine}

For our case study we use a simple idealised model 
of a register machine, which consists of a finite 
sequence of instructions, an instruction pointer 
and an infinite set of registers. 
The syntax of the machine's language is 
shown in~\cref{fig:regmachine-syntax}.

\begin{figure}[h!]
\[
\begin{array}{rclclll}
  P &\in& \Prog   &::=& i_1 \seq \dots \seq i_n           & \text{program}                 \\

  I &\in& \Inst     &::=&                                 & \text{instruction}             \\
              & &   & | & \assignInst{r}{v}               & \text{assign to register}      \\
              & &   & | & \opInst{\otimes}{r_1}{r_2}{r_3}  & \text{apply binary operation}  \\
              & &   & | & \jumpInst{r}{\ip{i}}            & \text{conditional jump}        \\
              & &   & | & \exitInst                       & \text{exit}                    \\
              & &   & | & \readInst{r}{x}                 & \text{read from memory}        \\
              & &   & | & \writeInst{x}{v}                & \text{write to memory}         \\

  r      &\in& \Reg     &   &                    & \text{thread-local register}   \\ 
  x      &\in& \Loc     &   &                    & \text{shared memory location}  \\ 
  v      &\in& \Z       &   &                    & \text{value}                   \\
  \otimes&\in& \BinOp   &   &                    & \text{binary operation}        \\
  \ip{i} &\in& \N       &   &                    & \text{instruction label}       \\ 

\end{array}
\] 
\caption{Syntax of the register machine}
\label{fig:regmachine-syntax}
\end{figure}

We first present the semantics of a single thread program.
Under this semantics, memory access instructions 
do not operate on shared memory
but rather produce a \emph{label}
denoting the side-effect of the operation 
(see \cref{fig:label-syntax}).
This encoding allows us to decouple 
the semantics of the register machine 
from a \emph{memory model}.

\begin{figure}[h!]
\[
\begin{array}{rclclll}

  \ell &\in& \Lab  &::=&                                & \text{}             \\
              & &  & | & \rlab{x}{v}                    & \text{read of value $v$ from location $x$}   \\
              & &  & | & \wlab{x}{v}                    & \text{write of value $v$ to location $x$}    \\

\end{array}
\] 
\caption{Syntax of Labels}
\label{fig:label-syntax}
\end{figure}

The semantics is given in the form of a
\emph{labelled transition system}~%
\footnote{As we have mentioned, in our \coq development we actually use 
the monadic encoding of the operational semantics. 
The labelled transition system can be derived from this encoding.}.
The state of the system consists of an instruction pointer
and a map from registers to their values, 
as shown in \cref{fig:regmachine-thrdstate}. 
The rules of the semantics are standard 
(see \cref{fig:regmachine-thrdsem}).

\begin{figure}[h!]
\[
\begin{array}{rclclll}

  s          &\in& \ThrdState    &::=& \tup{\ip{i}, \regmap} &                         \\
  \ip{i}     &\in& \N            &   &                    & \text{instruction pointer} \\
  \regmap    &\in& \Reg \fun \Z  &   &                    & \text{register mapping}   \\

\end{array}
\] 
\caption{Thread state of register machine}
\label{fig:regmachine-thrdstate}
\end{figure}

\begin{figure*}[t]

  \begin{subfigure}{0.5\linewidth}
  \begin{center}

  \AXC{$P[\ip{i}] =\ \assignInst{r}{v}$}
  \RightLabel{\texttt{Assign}}
  \UIC{$P \vdash \tup{\ip{i}, \regmap} \thrdstep{\eps} \tup{\ip{i+1}, \updmap{\regmap}{r}{v}}$}
  \DisplayProof
  
  \rulevspace
  
  \AXC{$P[\ip{i}] =\ \writeInst{x}{v}$}
  \RightLabel{\texttt{Store}}
  \UIC{$P \vdash \tup{\ip{i}, \regmap} \thrdstep{\wlab{x}{v}} \tup{\ip{i}+1, \regmap}$}
  \DisplayProof
   
  \rulevspace
  
  \AXC{$P[\ip{i}] =\ \readInst{r}{x}$}
  \RightLabel{\texttt{Load}}
  \UIC{$P \vdash \tup{\ip{i}, \regmap} \thrdstep{\rlab{x}{v}} \tup{\ip{i+1}, \updmap{\regmap}{r}{v}}$}
  \DisplayProof

  \end{center}
  \end{subfigure}
  %
  \begin{subfigure}{0.5\linewidth}
  \begin{center}
    
  \AXC{$P[\ip{i}] =\ \opInst{\otimes}{r_1}{r_2}{r_3}$}
  \AXC{$v = \appmap{\regmap}{r_2} \otimes \appmap{\regmap}{r_3}$}
  \RightLabel{\texttt{Binop}}
  \BIC{$P \vdash \tup{\ip{i}, \regmap} \thrdstep{\eps} \tup{\ip{i+1}, \updmap{\regmap}{r_1}{v}}$}
  \DisplayProof

  \rulevspace

  \AXC{$P[\ip{i}] =\ \exitInst$}
  \AXC{$\len(P) = n$}
  \RightLabel{\texttt{Exit}}
  \BIC{$P \vdash \tup{\ip{i}, \regmap} \thrdstep{\eps} \tup{\ip{n}, \regmap}$}
  \DisplayProof

  \rulevspace

  \AXC{$P[\ip{i}] =\ \jumpInst{r}{\ip{j}}$}
  \AXC{$\appmap{\regmap}{r} = 0$}
  \RightLabel{\texttt{CJump}$_z$}
  \BIC{$P \vdash \tup{\ip{i}, \regmap} \thrdstep{\eps} \tup{\ip{i+1}, \regmap}$}
  \DisplayProof
  
  \rulevspace
  
  \AXC{$P[\ip{i}] =\ \jumpInst{r}{\ip{j}}$}
  \AXC{$\appmap{\regmap}{r} \neq 0$}
  \RightLabel{\texttt{CJump}$_{nz}$}
  \BIC{$P \vdash \tup{\ip{i}, \regmap} \thrdstep{\eps} \tup{\ip{j}, \regmap}$}
  \DisplayProof

  \end{center}
  \end{subfigure}

  \caption{Semantics of register machine}
  \label{fig:regmachine-thrdsem}
\end{figure*}


