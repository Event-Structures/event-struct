\section{Background}

% In this section we give some background on event structures. 
There are several modifications of events structures, 
in this work we focus on \emph{prime event structures}~\cite{Winskel:86}.

\begin{definition} 
\label{es_def}
A prime event structure (PES) is a triple $(\eventSet, \caOrd, \cfRel)$, where
\begin{itemize}
  \item E is a set of events
  \item $\caOrd$ is a causality relation on E such that 
  \begin{itemize}
    \item $ (\eventSet, \caOrd) $ is a partial order;
    \item for every $e \in \eventSet$ its causality prefix ${\lceil e \rceil := \{ e' : e' \caOrd e \}}$ 
      is finite, \ie every event is caused by a finite set of events.
  \end{itemize}
  \item $\#$ is a conflict relation on $E$ such that:
  \begin{itemize}
    \item $\#$ is irreflexive and symmetric;
    \item it satisfies a \emph{hereditary} condition:
    $$ \forall e_1, e_2, e_3 \in E \ldotp \ e_1 \cfRel e_2 \land e_2 \caOrd e_3 \Rightarrow e_1 \cfRel e_3 $$
      That is, if two events are in conflict, then all their causal successors 
      are also have to be in confict. 
  \end{itemize}
\end{itemize}
\end{definition}

Prime event structure can encode multiple runs of a program.
Each individual run can be extracted as \emph{configuration}. 
In other words, configurations are used to model 
a history of computation up to a certain point.

\begin{definition}
A configuration of PES $(E, \caOrd, \cfRel)$ is a set of events $X \subseteq E$ such that
\begin{itemize}
  \item it is causally closed 
    $$ \forall e_1, e_2 \in E \ldotp e_2 \in X \land e_1 \caOrd e_2 \Rightarrow e_1 \in X $$
  \item and conflict-free 
    $$ \forall e_1, e_2 \in X \ldotp \neg (e_1 \cfRel e_2) $$
\end{itemize}
\end{definition}

We will also need a subclass of prime event structures --- 
\emph{confusion-free} event structures~\cite{Nielsen-al:1981}. 
In confusion-free structure the conflict relation is generated from 
a simpler \emph{primitive conflict} $\primcfRel$. 
It denotes pairs of events at which the compuation
diverges into two conflicting branches. 

\begin{definition}
Confusion-free event structures (CFES) is a triple $(E, \caOrd, \primcfRel)$, where
\begin{itemize}
  \item $\caOrd$ obeys the same laws as for prime event structure;
  \item $\primcfRel$ is a primitive conflict relation, such that
  \begin{itemize}
    \item it is irreflexive, symmetric and transitive;
    \item it does not violate hereditary condition:
    $$ \forall e_1, e_2, e_3 \in E \ldotp 
       e_1 \primcfRel e_2 \land e_2 \caOrd e_3 \Rightarrow \neg (e_1 \primcfRel e_3) $$
    \item it is \emph{choice-local}
  \end{itemize}
    $$ \forall e_1, e_2, e_3 \in E \ldotp e_1 \caOrd e_2 \land e_2 \primcfRel e_3 \Rightarrow e_1 \caOrd e_3 $$
\end{itemize}
\end{definition}

From a confusion-free event structure one can construct a prime event structure
by defining $\cfRel$ as follows:

$$ e_1 \cfRel e_2 \iff 
   \exists e'_1, e'_2 \ldotp e'_1 \primcfRel e'_2 \land
   e'_1 \caOrd e_1 \land e'_2 \caOrd e_2 $$

It can be shown that the conflict relation defined in this way 
satisfies the axioms of prime event structure.

% The following simple facts about prime event structures:

% \begin{lemma}
%   A set $\lceil e \rceil$ is conflict-free for any $e \in E$.
% \end{lemma}

% And a more general one:

% \begin{lemma}
%   A set $\lceil e \rceil$ is a configuration for any $e \in E$.
% \end{lemma}

