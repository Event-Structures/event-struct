\section{Overview of Our Library}

In this section, we sketch the design principles of our library. 

We build our mechanisation on top of the \mathcomp~\cite{Mahboubi-Tassi:MATHCOMP17} library 
which is an extensive and coherent repository of formalized mathematical theories,
whose implementation is based on the \ssreflect~\cite{Gonthier-al:SSR2016} extension of the \coq system.
By using \mathcomp, we draw on the large corpus of already formalised algorithms and mathematical results:
its core modules feature support for a range of useful data structures,
e.g. numbers, sequences, finite graphs,
and also interfaces: types with decidable equality, subtypes, finite types, and so on.

We also use the \emph{small-scale reflection} 
methodology~\cite{Gonthier-Assia:SSR2010, Gonthier-al:SSR2016}, 
a key ingredient of \ssreflect. 

The small-scale reflection approach is based on 
the pervasive use of symbolic representations \emph{intermixed}
with logical ones within the confines of the same proof goal,
as opposed to large-scale reflection which does not allow such mixing.
Symbolic representations are connected to the corresponding logical ones
via user-defined \emph{reflect predicates}.
The symbolic representation can be manipulated 
by the computational engine of the language, 
allowing the user to automate low-level routine 
proof management by using various decision 
and simplification procedures.
Whenever the user needs to guide the proof 
they can switch to the logical representation
and perform some proof steps manually. 

To achieve better automation and e.g. get \emph{proof irrelevance}
for free, one is encouraged
to use \emph{decision} procedures whenever possible.
For example, in the context of our library, 
we encode the binary relations of the event structures
as decidable \texttt{bool}-valued relations, 
\ie $\caOrd, \cfRel : E \fun E \fun \texttt{bool}$,
as opposed to \emph{propositional} 
relations of type $E \fun E \fun \texttt{Prop}$. 

In fact, binary relations are omnipresent in our formalization.
In order to further automate the reasoning about them, 
we have chosen to use the \relationalgebra%
\footnote{\url{https://github.com/damien-pous/relation-algebra}}~%
\cite{Pous-ITP2013} package 
which provides a number of tactics to solve goals using
decision procedures for the theory of lattices  
and \emph{Kleene Algebras with Tests}~\cite{Kozen:TOPLAS:1997}.

We also favour the computational encoding of semantics. 
Similar to the recent related works on mechanisation 
of operational semantics~\cite{Xia-al:POPL2019, Letan-al:CPP2020, Affeldt-al:ICMPC2019}, 
we encode the semantics as monadic interpreters.  
This allows us to extract~\cite{Letouzey:CCE2008} 
the semantics as a functional program and run it. 
We believe that the possibility to run the semantics 
is a very useful feature, as it allows 
to debug the formal semantics
and helps to develop better intuition about it.

To facilitate computable semantics, we define a subclass of
finitely supported event structures as a finite sequence of events
combined with a finitely supported function which enhances events with 
additional information, such as their labels, causality predecessors, \etc 
Encoding finitely supported functions is not a trivial endeavor in
a proof assistant and for this task we use the \finmap~%
\footnote{\url{https://github.com/math-comp/finmap}}
library which is an extension of \mathcomp providing finite sets and
finite maps on types with a choice operator (rather than on finite types).

Finally, to encode the algebraic hierarchy of various classes of event structures
we use yet another feature of \mathcomp --- 
\emph{packed classes}~\cite{Garillot-al:ICTPHOL2009},
which is a design pattern providing multiple inheritance,
maximal sharing of notations and theories,
and automated structure inference. % and it also deals with performance issues.

% TODO
% The relation-algebra package and lattice theory
% well-foundedness and transitive closure for boolean relations
% identType and how to fight quadratic time complexity
